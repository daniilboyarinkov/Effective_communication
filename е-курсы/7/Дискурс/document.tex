\documentclass[a4paper,14pt,titlepage]{article}

% Кодировка
\usepackage[T2A]{fontenc}
\usepackage[utf8]{inputenc}
% языки
\usepackage[english,russian]{babel}
\usepackage{amsmath,amsthm}
\usepackage{amssymb}
\usepackage[14pt]{extsizes}

\usepackage{ragged2e}
% Выравнивание текста по ширине \justifying
\newcommand{\jj}{\righthyphenmin=20 \justifying}
% Подписи для рисунков
\usepackage{caption}

\usepackage{extsizes}
% Ссылки
\usepackage{hyperref}
% цвета ссылок
\usepackage{xcolor}
\hypersetup{colorlinks=true, urlcolor=gray, linkcolor=gray}

% эпиграф и цитаты
\usepackage{epigraph}
\usepackage{csquotes}

% графика
\usepackage[active]{srcltx}
\usepackage{graphicx}
\graphicspath{{images/}}
\DeclareGraphicsExtensions{.pdf,.png,.jpg}

% Отступы
\usepackage{geometry}
\geometry{left=2cm}
\geometry{right=2cm}
\geometry{top=2cm}
\geometry{bottom=2cm}

\usepackage{tabularx}
\renewcommand{\arraystretch}{1.8}

% Принудительное переопределение размера section и susection в 14pt размер
%\usepackage{sectsty}
%\sectionfont{\fontsize{14}{14}\selectfont}
%\subsectionfont{\fontsize{14}{14}\selectfont}

% Начальная страница
\title{Дополнительная информация из материалов, представленных на е-курсах \\ Дискурс}
\author{Бояринков Даниил Владимирович \\ Группа: КИ21-22Б}
\date{22 Октября 2021}

\begin{document}
	\maketitle
	\setcounter{page}{2}
	\centering
	\tableofcontents
	\newpage
	
	\section{Дискурс}
	\textbf{Дискурс} -- это сложное коммуникативное явление, включающее связный текст в совокупности с экстралингвистическими факторами. То есть, это текст, взятый в событийном контексте.
	
	\subsection{Типы дискурса: }
	\justifying
	\begin{enumerate}
		\item Персональный
		\item Институциональный
	\end{enumerate}
	Институциональный представляет собой устойчивую систему статусно-ролевых отношений, сложившуюся в коммуникативном пространстве жизнедеятельности определенного социального института, в рамках которой осуществляется властные функции символического принуждения в форме нормативного предписания и легитимации определенных способов мироведения, мирочувствования, векторов ценностных ориентаций и моделей поведения.
	
	\section{Статус}
	\textbf{Статус} -- это место субъекта в социальной иерархии.
	\section{Жанры институционального дискурса: }
	\begin{itemize}
		\item политический
		\item дипломатический
		\item административный
		\item научный
		\item юридический
		\item педагогический
		\item медицинский
		\item деловой
		\item военный
		\item религиозный
		\item мистический
		\item рекламный
		\item сценический
		\item массово-информационный
	\end{itemize}	
	
\end{document}